\documentclass[11pt]{scrartcl}

\usepackage{hyperref}

\title{DECIND Project Whitepaper}
\subtitle{Decentralized Index Funds}
\date{Version 0.3.0 - \today}
\author{Louis FRADIN}


\begin{document}
    \maketitle
    \section{Overview}

    \subsection{Initial Problem}
    Investing in the cryptographic finance world is technically hard, time consuming,
    or can be costly in terms of fees. Solutions to invest easily on 
    cryptocurrencies exist, but unfortunately they require users to trust a company with
    their funds and customers are loaded for using their products. Sometimes, the 
    minimum amount required for these products is just too high for the common
    people. The vast majority of people cannot invest easily on the promising
    decentralized finance, lacking trustless, ease of use and low entry ticket
    products.

    \subsection{Proposed Solution}
    The traditional finance world created a popular solution for this problem:
    Exchange Traded Funds or ETFs. These funds replicate stock indexes and are
    gaining in popularity as they perform better than active trading on the long 
    term : \textit{time in market beats timing the market}. Also, they are simple,
    low on fees and automatable.

    The idea behind the Decentralized Index Funds Project is to use decentralized finance exchanges
    with smart contracts to replicate indexes on most valued cryptocurrencies.
    The project aimed to have the lowest fees, be trustless, and
   decentralized. The final goal is to give access to an investment product 
    in the crypto world that is cheap and simple, for everybody. For this goal 
    to become a reality, not only this needs well-designed products, but 
    also a need for simple investment tools for non-tech people.

    \section{Index Fund Token}

    \subsection{Assets investment and tokens creation}
    The user can buy compatible cryptographic assets in exchange markets and
    convert them through the smart contract to the \textit{IFT}.
    
    \textit{IFTs} will be created to represent the asset value,
    when it enters the fund, as user investment. The \textit{IFTs amount} 
    generated by the smart contract corresponds to the asset value in \textit{IFT} 
    price.

    \paragraph{Example}
    A user will invest \$100 worth of \textit{ETH} in the fund
    and will get \$100 of freshly created IFT in return. If the \textit{IFT} is
    valued at \$10, the user will get 10 \textit{IFT}. 
    
    \subsection{Assets liquidation and tokens destruction}
    When users wants to get back their investments, they can sell their \textit{IFTs} 
    on exchange markets or liquidate by smart contract against any wanted cryptographic 
    asset available.

    The \textit{IFT} retrieved will be exchanged for a wanted 
    cryptographic asset amount at the \textit{IFT} price, as it represents the
    \textit{UAV}. \textit{IFTs} retrieved by the contract will be burnt as the
    asset as left the fund to keep the IFT value consistence.

    Following the previous example, the user wants to get his investments back.
    The \textit{UAV} has increased, and the \textit{IFT} price is now at \$20. 
    The user will liquidate his 10 \textit{IFTs} for \$200 worth of cryptographic
    assets.

    \subsection{Conversion Costs}
    Conversion cost to required assets in order to keep the index balanced will 
    be ensured by those investors, and not by the fund to avoid other investors
    penalization. This conversion should take place when adding assets to the
    or withdrawing them. As an incentive for investors to introduce
    the right coin, if they provide the correct asset, they won't be loaded as 
    there is no conversion to do. This will also be positive for the index fund 
    as the need for asset balancing is diminished.

    \subsection{Arbitrage Trading}
    \textit{IFT} is an exchangeable token, so it can be bought on any exchange 
    market that lists the coin. To avoid the \textit{IFT} value deviation from 
    its real value, there is a mechanism that act as an incentive to keep the 
    right price.

    As the assets on the fund will increase/decrease in value and quantities, the 
    value on exchange markets will deviate between the token and the underlying assets.
    This gap between prices is profitable and will be used as an incentive for
    traders to fill the gap between assets market value, \textit{UAV}, and \textit{IFT}
    market value. These traders doing arbitrage between exchange markets and smart contracts
    are called \textit{arbitrage traders}.

    \paragraph{Example}
    A user just withdrawn his investments, there are now fewer assets in the fund,
    driving the \textit{UAV} lower. On exchange markets, the token value is now higher
    than the \textit{UAV}. If a trader sells his token funds at a high price in
    exchange markets and exchange the same token amount through the smart contract for a
    lower price, then the operation is profitable and acts as an incentive. 
    The opposite is also true.

    \subsection{Governance Token}
    \textit{IFT} holders have an amount of \textit{IFT} representing the
    index fund assets percentage that they detain, compared to the
    total \textit{IFT} supply. Also, as the assets are detained by \textit{IFT} holders,
    they have investments at stake : the more \textit{IFT} you have, the more
    you have at stake. \textit{IFT} holders are best placed to decide on matters of
    index fund decisions.
    
    \section{Index Fund Pool}

    \subsection{Description}
    The Index Fund Pool, or \textit{IFP}, is the pool that contains the assets.
    The pool will automatically balance to follow the index fund strategy, when 
    some conditions are encountered, via blockchain liquidity pools.

    \paragraph{Example}
    The index is following a strategy that holds the top X cryptographic assets,
    weighted by their market capitalization. If one of the assets is over or under 
    represented by a certain amount or after a certain amount of time, the pool 
    can automatically rebalance to match the targeted allocation.

    \subsection{Funds use}
    Through decentralized finance technological advancements, the underlying 
    assets usage can be considered for different purposes (as lending, or any other), 
    on the condition that there are no assets loss risks. 

    \subsection{Cross-chain funds federation}
    As some of the interesting functionnalities can be located on some blockchains
    and not others, or if there is lower fees, and more, it can be interesting to
    have the index placed on multiple blockchains.

    Now that cross-chain technology is being developped, it is possible to operate 
    the main token, the \textit{IFP}, on the cross-chain blockchain (as
    Fusion\footnote{Fusion (\url{https://fusion.org})}) and some sub-tokens, or \textit{subIFP},
    on specific blockchains to dispatch the funds. As the coherency between main token and 
    sub-tokens can deviate, the incentive for \textit{arbitrage trading} will close the gap.

    \paragraph{Example} 
    The \textit{IFP} is located on the Fusion network, or \textit{FSN}, and the
    \textit{subIFPs} are located on Ethereum and Fantom network.
    The \textit{subIFPs} are named \textit{seIFP} on Ethereum network and
    \textit{sfIFP} on Fantom network. Those two sub-funds contain differents coins for
    different reasons. For the sake of this example, the \textit{IFP} is
    composed of the two \textit{subIFPs} at equal values (\textit{IFP} = 0.5
    \textit{seIFP} and 0.5 \textit{sfIFP}). Also \textit{seIFP} = \$2 and
    \textit{sfIFP} = \$2, so \textit{IFP} = \$4.

    Now Alice buy some \textit{IFP} and its price rises to \$6. This means that
    the \textit{subIFPs} are valued \$3 on the Fusion network. However, on the 
    Ethereum and Fantom network, \textit{subIFPs} are still at \$2. That is why 
    Bob, an arbitrage trader, will buy \textit{subIFPs} on Ethereum and Fantom 
    networks to transfer them to the Fusion network and sell them. On Ethereum 
    and Fantom network, the price will rise (\$2,5 by example), on the Fusion 
    network, the price will fall (\$2,5 by example), and the balance will be 
    restored.

    \subsection{Development sustainability}
    As developers are invited, as everyone, to invest in the index fund, they can 
    work for the Index Fund by ensuring or improving its profitability. Their 
    work will be directly rewarded by this profitability, through their investment.
    Therefore, there is no need for more incentive, limiting fees.

    However, if there is a need for critical/sustained development or even bug
    bounties and so forth, a part of the foundation allowance could be used as
    an incentive or a reward for developers (see more on the foundation section).

    \section{Index Fund Board}

    \subsection{Governance}
    As index fund holders have investments represented by the \textit{IFT},
    they tend to make the best decisions regarding the index fund profitability.
    So, by holding their \textit{IFT}, holders will be rewarded with
    a governance right. 

    As underlying assets are the only index fund possessions, they 
    literally represent the index fund organization shares.
    With the same idea as a company’s shares, \textit{IFT} gives 
    every holder a right to vote proportionally to their \textit{IFT} amount
    (more information on the index fund organization section).
    
    \subsection{Holding incentive}
    As \textit{IFT} earners have their investments at stake,
    it is better for them to keep this governance token to make the best decision
    for their index locked assets.
    
    Also not selling \textit{IFT} avoids letting
    other people making decisions by selling their governance token to them.
    But even if they prefer to sell their \textit{IFTs}, buyers should also make
    the right decisions or they will lose their \textit{IFTs} value, as they 
    are now the investors.

    \subsection{Decision process}
    Decisions will be made as a \textit{DAO}\footnote{Decentralized Autonomous Organization
        (\url{https://en.wikipedia.org/wiki/Decentralized_autonomous_organization})}
    and will be submitted to vote. \textit{IFT} holders can use their funds to
    weigh their vote in the decision process. 

    \paragraph{Example}
    Next developments and finished developments deployments will be submitted to 
    the DAO members for approval before being executed.

    \section{Index Fund Foundation}
    As the project needs external services for the ecosystem to be complete or
    for people responsible for the project running.
    
    \subsection{Role}
    The foundation should act as an executive of the board decisions and should use 
    its available funds (describe later) to accomplish its mission. It is free to 
    act for the index fund right development and for the foundation sustainment, but 
    should always respect the board decisions. The fund should also be transparent
    on its activities and make all available documents readable by the 
    board.

    \subsection{Allowance}
    The foundation purpose will need funds in order to operate.
    Consequently, an allowance can be applied to the fund to help this 
    effort. Obviously, this allowance must be as little as possible to be the less of a 
    burden as possible. If the funds detained by the foundation are enough and
    if the capital gain is not enough to diminish these funds with the necessary
    spending, then the allowance will be disabled.
    
    That is why the allowance will be generated in function of its revenue and
    spending, with a max value (0,12\%/year). This means that bigger is the fund,
    lower is the allowance percentage and if the fundation is well managed and the 
    market is prosperous, then it should be zeroed rapidly.

    This allowance will be generated inside the protocol by generating new \textit{IFTs}, 
    respecting the allowance. As the \textit{IFT} value is linked to the \textit{UAV}, 
    when the first one quantity increase without the second one to change, it will 
    decrease the \textit{IFT} value, as a dilution. This resulting in from all fund participants to the allowance, without having 
    to pay for assets exchange. If this allowance is really low faced to the 
    fund value, it will be nearly imperceptible (e.g. 0.12\% per year by example).
\end{document}